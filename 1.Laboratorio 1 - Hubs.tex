\documentclass{article}
\usepackage[utf8]{inputenc}
\usepackage{graphicx}
\usepackage{amsmath}
\usepackage{listings}
\usepackage{caption}
\usepackage[a4paper, margin=1in]{geometry} 
\setlength{\parskip}{1em}
\usepackage{float}
\usepackage{hyperref}
\usepackage{booktabs} % Opcional: para mejorar la estética de las tablas
\renewcommand{\thesection}{\arabic{section}.}


\title{Laboratorio 1 - Hubs \\ Redes I}
\author{Giovani Bavera \\ giovani.bavera@gmail.com}
\date{August 30, 2024}

\begin{document}

\maketitle

\section{Conexión de dos dispositivos}

En los casos de las PCs se pueden ver que las IPs terminan en numero bajos:

\begin{itemize}
    \item PC1 -- 192.168.0.1
    \item PC2 -- 192.168.0.2
\end{itemize}

Mientras que, en el caso del servidor, el numero de su IP termina en un numero mayor: 

\begin{itemize}
    \item Server -- 192.168.0.100
\end{itemize}

Podemos notar que los servidores, al ser dispositivos que están constantemente encendidos, requieren de una IP estática y segura, para eso se le asigna una de mayor terminación porque son direcciones menos propensas a cambios o utilización de otro dispositivo. Las IPs de las PCs, así como la de otros dispositivos finales, al encenderse y apagarse utilizan una IP dinámica y de menor terminación siendo que pueden tomar las direcciones lógicas entre ellas. 
En cuanto a las direcciones físicas (MACs), cada dispositivo posee una propia que fue asignada por el fabricante de la NIC.

\section{Conexión de dos dispositivos}

\begin{table}[h]
    \centering
    \begin{tabular}{|c|c|c|}
        \hline
        \textbf{Dispositivo} & \textbf{IP} & \textbf{MAC} \\
        \hline
        PC1 & 192.168.0.3 & 255.255.255.0 \\
        \hline
        PC2 & 192.168.0.1 & 255.255.255.0 \\
        \hline
        PC3 & 192.168.0.4 & 255.255.255.0 \\
        \hline
        Server & 192.168.0.100 & 255.255.255.0 \\
        \hline
    \end{tabular}
    \caption{Tabla de dispositivos con sus direcciones IP y MAC}
    \label{tab:devices}
\end{table}

En esta tabla se observa las direcciones físicas y lógicas de los distintos dispositivos. Como las PCs están configuradas en DHCP, las IPs van a ser asignadas dinámicamente. Es decir, una PC puede tomar la IP de otra si se apaga ya que pasa a estar disponible y cuando vuelva a encenderse la otra tomar una IP que este disponible. 
En este caso que tenemos un HUB, para ver las direcciones lógicas de cada End Device tendremos que acceder manualmente a cada consola. Si hubo una comunicación previa de un dispositivo a todos los demas, es posible ver las direcciones con el comando \textit{arp -a}. 

\section{Dominios de colisión}

\begin{figure}[H]
    \centering
    \includegraphics[width=0.35\linewidth]{1HUB-4ED.png}
    \caption{}
    \label{fig:enter-label}
\end{figure}

Teniendo la situación de un Hub con 4 dispositivos (figure 1) se produce una colisión cuando mas de un mensaje tienen una coincidencia temporal, provocando una falla de comunicación

\begin{figure}[H]
    \centering
    \includegraphics[width=0.8\linewidth]{3hubs.png}
    \caption{}
    \label{fig:enter-label}
\end{figure}           

Mismo caso en el de la figura 2. La colisión tarde o temprano se va a producir en un único HUB (sea el 0,1 o 2) cuando los mensajes de diferentes dispositivos se encuentren provocando el fallo. Posteriores mensajes enviados van a fallar consecuentemente. 

Esto se puede evitar al editar el tiempo de salida de los mensajes, introduciendo un retraso de salida para que no haya un "choque" entre ellos y pueda funcionar todo normalmente. O un algoritmo que ayude en estos fallos. 


\section{Referencias}

\textbf{Github:} \url{https://github.com/GioBavera/redes1.git}
\end{document}
